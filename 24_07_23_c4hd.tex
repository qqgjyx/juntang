\documentclass[journal,12pt,onecolumn, draft]{IEEEtran}

\usepackage{amsmath}  % For mathematical equations
\usepackage{graphicx} % For including graphics
\usepackage{cite}     % For handling citations
\usepackage{hyperref} % For hyperlinks in the document
\usepackage{bm}
\usepackage{xcolor}   % For text colors

\begin{document}
\begin{table}[h!]
\centering
\caption{Summary of hyperspectral datasets}
\label{tab:datasets}
\begin{tabular}{lcccccc}
\hline
\textbf{Dataset} & \textbf{\#x} & \textbf{\#y} & \textbf{\#\bm{$\lambda$}} & \textbf{\bm{$\lambda$} (nm)} & \textbf{\#C} & \textbf{Res. (m)} \\
\hline
Indian Pines\footnotemark[1] & 145 & 145 & 220 & 400--2500 & 16 & 20 \\
Salinas\footnotemark[2] & 512 & 217 & 204 & 400--2500 & 16 & 3.7 \\
Salinas-A\footnotemark[2] & 86 & 83 & 204 & 400--2500 & 6 & 3.7 \\
Pavia\footnotemark[3] & 1096 & 1096 & 102 & 430--860~ & 9 & 1.3 \\
Pavia University\footnotemark[3] & 610 & 610 & 103 & 430--860~ & 9 & 1.3 \\
\hline
\end{tabular}
\end{table}

\begin{table}[h!]
\centering
\caption{Performance of unsupervised clustering methods on hyperspectral datasets}
\label{tab:clustering_methods}
\begin{tabular}{lcccc}
\hline
\textbf{Dataset} & \textbf{GMM}\footnotemark[4] & \textbf{DBSCAN}\footnotemark[5] & \textbf{HDBSCAN}\footnotemark[6] & \textbf{Spectral}\footnotemark[7] \\
\hline
Pines & \begin{tabular}{c|c} \textcolor{red}{23.61} & \textcolor{blue}{16.36} \\ \hline \textcolor{black}{0.0889} & \textcolor{orange}{0.1231} \end{tabular} & \begin{tabular}{c|c} \textcolor{red}{40.26} & \textcolor{blue}{1.68} \\ \hline \textcolor{black}{0.0356} & \textcolor{orange}{0.1665} \end{tabular} & \begin{tabular}{c|c} \textcolor{red}{50.98} & \textcolor{blue}{5.97} \\ \hline \textcolor{black}{0.0110} & \textcolor{orange}{0.1860} \end{tabular} & \begin{tabular}{c|c} \textcolor{red}{32.67} & \textcolor{blue}{15.86} \\ \hline \textcolor{black}{0.1138} & \textcolor{orange}{0.1355} \end{tabular} \\
\hline
Salinas & \begin{tabular}{c|c} \textcolor{red}{37.03} & \textcolor{blue}{14.46} \\ \hline \textcolor{black}{0.2227} & \textcolor{orange}{0.3177} \end{tabular} & \begin{tabular}{c|c} \textcolor{red}{39.73} & \textcolor{blue}{7.10} \\ \hline \textcolor{black}{0.1969} & \textcolor{orange}{0.3173} \end{tabular} & \begin{tabular}{c|c} \textcolor{red}{3.37} & \textcolor{blue}{5.56} \\ \hline \textcolor{black}{0.0002} & \textcolor{orange}{0.2699} \end{tabular} & \begin{tabular}{c|c} \textcolor{red}{50.86} & \textcolor{blue}{16.85} \\ \hline \textcolor{black}{0.1621} & \textcolor{orange}{0.2338} \end{tabular} \\
\hline
SalinasA & \begin{tabular}{c|c} \textcolor{red}{21.95} & \textcolor{blue}{15.70} \\ \hline \textcolor{black}{0.1587} & \textcolor{orange}{0.5479} \end{tabular} & \begin{tabular}{c|c} \textcolor{red}{16.99} & \textcolor{blue}{3.89} \\ \hline \textcolor{black}{0.1399} & \textcolor{orange}{0.4698} \end{tabular} & \begin{tabular}{c|c} \textcolor{red}{28.78} & \textcolor{blue}{21.39} \\ \hline \textcolor{black}{0.2164} & \textcolor{orange}{0.5152} \end{tabular} & \begin{tabular}{c|c} \textcolor{red}{25.27} & \textcolor{blue}{25.53} \\ \hline \textcolor{black}{0.0545} & \textcolor{orange}{0.4621} \end{tabular} \\
\hline
Pavia & \begin{tabular}{c|c} \textcolor{red}{13.64} & \textcolor{blue}{11.51} \\ \hline \textcolor{black}{0.0055} & \textcolor{orange}{0.0585} \end{tabular} & \begin{tabular}{c|c} \textcolor{red}{79.33} & \textcolor{blue}{8.78} \\ \hline \textcolor{black}{0.1083} & \textcolor{orange}{0.4116} \end{tabular} & \begin{tabular}{c|c} \textcolor{red}{81.09} & \textcolor{blue}{9.09} \\ \hline \textcolor{black}{0.0000} & \textcolor{orange}{0.0313} \end{tabular} & \begin{tabular}{c|c} \textcolor{red}{85.13} & \textcolor{blue}{25.94} \\ \hline \textcolor{black}{0.4747} & \textcolor{orange}{0.4135} \end{tabular} \\
\hline
PaviaU & \begin{tabular}{c|c} \textcolor{red}{26.86} & \textcolor{blue}{19.88} \\ \hline \textcolor{black}{0.0315} & \textcolor{orange}{0.0101} \end{tabular} & \begin{tabular}{c|c} \textcolor{red}{77.65} & \textcolor{blue}{17.10} \\ \hline \textcolor{black}{0.0395} & \textcolor{orange}{0.1594} \end{tabular} & \begin{tabular}{c|c} \textcolor{red}{79.22} & \textcolor{blue}{9.10} \\ \hline \textcolor{black}{0.0008} & \textcolor{orange}{0.0304} \end{tabular} & \begin{tabular}{c|c} \textcolor{red}{58.01} & \textcolor{blue}{26.18} \\ \hline \textcolor{black}{0.0817} & \textcolor{orange}{0.0232} \end{tabular} \\
\hline
& \multicolumn{4}{c}{\textbf{Metrics: \textcolor{red}{OA (\%)}, \textcolor{blue}{AA (\%)}, \textcolor{black}{Kappa}, \textcolor{orange}{ARI}}} \\
\hline
\end{tabular}
\end{table}

% Footnotes for the datasets
\footnotetext[1]{Indian Pines dataset: \href{https://aviris.jpl.nasa.gov/dataportal/}{https://aviris.jpl.nasa.gov/dataportal/}}
\footnotetext[2]{Salinas and Salinas-A datasets: \href{https://aviris.jpl.nasa.gov/dataportal/}{https://aviris.jpl.nasa.gov/dataportal/}}
\footnotetext[3]{Pavia datasets: Data provided by Paolo Gamba, University of Pavia, available at \href{https://tlclab.unipv.it/}{https://tlclab.unipv.it/}}
% Footnotes for the acronyms
\footnotetext[4]{GMM: Gaussian Mixture Model}
\footnotetext[5]{DBSCAN: Density-Based Spatial Clustering of Applications with Noise}
\footnotetext[6]{HDBSCAN: Hierarchical Density-Based Spatial Clustering of Applications with Noise}
\footnotetext[7]{Spectral: Spectral Clustering}

% Code availability
\textbf{Code Availability:} The code used for clustering analysis on hyperspectral datasets is available at \href{https://gitlab.oit.duke.edu/jw853/clustering4hsi}{https://gitlab.oit.duke.edu/jw853/clustering4hsi}. This repository provides scripts and instructions for replicating the clustering results reported in this paper.

\end{document}